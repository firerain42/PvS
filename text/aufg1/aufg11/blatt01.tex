\documentclass{article}

\usepackage{enumerate}

\begin{document}

\section*{\center Programmierung von Systemen}
Lukas Petrich (754289)\hfill 	Blatt 1\\
Tutor: Michael Wendt \\[5ex]

\large{Aufgabe 1-1:}
\begin{enumerate}
	\item \hfill\\ \begin{tabular}{|l||c|c|c|}
		\hline 
		\rule[-1ex]{0pt}{2.5ex}  & hungry & food & legs \\ 
		\hline \hline
		\rule[-1ex]{0pt}{2.5ex} Class Animal & Ja & Ja & Nein \\ 
		\hline 
		\rule[-1ex]{0pt}{2.5ex} Class Bird & Nein & Ja (selbes Package) & Ja \\ 
		\hline 
		\rule[-1ex]{0pt}{2.5ex} Class Eagle & Nein & Ja (selbes Package) & Ja \\ 
		\hline 
		\end{tabular} 
	
	\item Overriding: \texttt{Animal.eat()} $\rightarrow$ \texttt{Bird.eat()}\\
	 Overloading: \texttt{Eagle.fly()} bzw. \texttt{Eagle.fly(int height)}
	\item \hfill\\ \begin{tabular}{|c|l|}
		\hline 
		1 & in Ordnung \\ 
		\hline 
		2 & 32 \\ 
		\hline 
		3 & funktioniert nicht, da Eagle Bird erweitert. \\ 
		\hline 
		4 & in Ordnung \\ 
		\hline 
		5 & 32 \\ 
		\hline 
		6 & in Ordnung \\ 
		\hline 
		7 & funktioniert nicht, da \texttt{fly(int height) private} ist  \\ 
		\hline 
		\end{tabular} 
\end{enumerate}	


\end{document}